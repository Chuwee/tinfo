\subsubsection{Desafíos}
\begin{enumerate}
    \item \textbf{Ruido}. Las comunicaciones de tráfico aéreo entre pilotos y controladores utilizan \textbf{amplitud modulada} (AM) \cite{WhatRadioEquipmentPilotsUseToCommunicateWithATC2018}. Esto provoca susceptibilidad a ruido, estática y solapamiento, lo que interfiere con la capacidad de un componente ASR para traducir eficientemente los audios a texto. Los receptores VHF (\textit{Very High Frequency}) son especialmente susceptibles a un nivel demasiado bajo de SNR (\textit{Signal to noise}) \cite{ZuluagaGomez2023Lessons} \cite{ChenKopaldMaTarakanWei2021ATCSpeechRecognition}.
    \item \textbf{Diferencias en el lenguaje, acentos y pronunciaciones} de los controladores. Ocasionalmente, los controladores de distintas regiones prefieren hablar su lengua nativa con pilotos que compartan la región (por ejemplo, pilotos de Iberia pueden hablar en español con controladores de torres de control españolas). La diferencia fonética en acentos (inglés contra indio, por ejemplo) puede suponer un desafío adicional \cite{SimpleFlying_NonEnglishATCPilotCommunicationsGuide} \cite{wee2024adapting}.
    \item \textbf{Fraseología y jerga aeronáutica} que no es convencional respecto al lenguaje hablado diariamente. Términos como \textit{callsign}, \textit{runway}, \textit{climb} pueden no coincidir con el conjunto del lenguaje objetivo de los modelos convencionales \cite{ZuluagaGomez2024ATCO2} \cite{Fan2024CustomizationASR}.
    \item \textbf{Puntos críticos del lenguaje} como los \textit{callsign} o las magnitudes de las instrucciones proporcionadas presentan puntos únicos de fallo. Un fallo en la magnitud de las instrucciones o a quién van dirigidas compromete por completo la maniobra que se quiera comunicar \cite{ZuluagaGomez2021Contextual} \cite{ATCO2_EndToEndCallsignRecognition2021}.
\end{enumerate}