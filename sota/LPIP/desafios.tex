\subsubsection{Desafíos}
\begin{enumerate}
    \item \textbf{Fraseología variable}. Un controlador aéreo puede plantear la misma instrucción a una aeronave de distintas maneras (aunque exista el estándar OACI \cite{eurocontrol_icao_nodate}, de tal forma que muchas instrucciones pueden tener la misma identidad. Por ejemplo:
    \begin{center}
        \texttt{IBERIA292 maintain three five zero zero}
        
        \texttt{IBERIA292 climb to three five zero zero}
    \end{center}
    El parseador óptimo debería interpretar ambas instrucciones como una petición de colocarse a una altura de 3500 pies.
    \item \textbf{Errores de transcripción} del módulo ASR pueden hacer difícil la resiliencia de todo el pseudopiloto. Un caso de ejemplo es el siguiente:
    \begin{center}
        \texttt{IBERIA292 climb to three five oh oh}
    \end{center}
    que quiere decir lo mismo de antes: es una instrucción para colocarse a 3500 pies. Este tipo de fraseología alternativa es crítica pues puede darse en \textit{callsigns} o en valores de instrucción. Un fallo en la interpretación de cualquiera de estos supone un punto único de fallo.
    \item \textbf{Identificación precisa de \textit{callsigns}} \cite{callsign_recognition}, que contienen números y pueden confundirse con magnitudes de instrucciones. Por ejemplo, podríamos pensar en la siguiente transcripción del ASR:
    \begin{center}
        \texttt{IBERIA two nine two zero heading}
    \end{center}
    Un posible parseador de magnitudes podría no saber dónde acaba el \textit{callsign} y dónde empieza la magnitud de la instrucción.
    \item \textbf{Múltiples parámetros en la misma instrucción}, como se ilustra que existen en \cite{Zuluaga‑Gomez2023Virtual}. Podríamos pensar en la siguiente transcripción:
    \begin{center}
        \texttt{climb to FL350, turn left heading 090, speed 250}
    \end{center}
    donde tenemos tres parámetros (altura, rumbo y velocidad) para una sola aeronave. 
    \item \textbf{Estructuras temporales encadenadas}, donde se proporcionan instrucciones secuenciales a aeronaves. Es esencial mantener la cohesión temporal en las instrucciones:
    \begin{center}
        \texttt{climb to FL350, after reaching point X descend to 3000}
    \end{center}
    \item \textbf{Normalización}, de las unidades y parámetros de instrucciones. \texttt{FL350}, \texttt{Flight Level three five zero}, \texttt{three five zero} son maneras de decir 35000 pies, y deben interpretarse como tal.
    \item \textbf{Resiliencia ante espontaneidad del discurso}, ya que instrucciones que involucran correcciones de ellas mismas deben ser interpretadas como si fueran instrucciones normales.
\end{enumerate}