\documentclass{article}
\usepackage{csquotes}
\usepackage[backend=biber]{biblatex}
\addbibresource{references.bib}
\usepackage{graphicx} % Required for inserting images
\usepackage{geometry}
\usepackage{amsmath}
\usepackage{amssymb}
\usepackage{booktabs}
\usepackage[spanish]{babel}
\usepackage{xcolor}
% beta 
% \pagecolor[rgb]{0,0,0} \color[rgb]{1,1,1}
\DeclareUnicodeCharacter{202F}{\,}

% URL breaking configuration (biblatex already loads url package)
\def\UrlBreaks{\do\/\do-\do_\do.\do=\do?\do&}
\urlstyle{same}

% Better line breaking to handle overfull hbox
\sloppy

\title{Preliminar: TFG}
\author{Ignacio de Miguel}
\date{September 2025}

\begin{document}

\maketitle

\newpage

\section{Planteamiento inicial}
El modelo que proponemos consiste en una \textit{pipeline} de cuatro elementos.
\begin{enumerate}
    \item El elemento de \textbf{captura y transcripción de voz}. En este entorno, tradicionalmente se le conoce como \textbf{ASR} (\textit{Automatic Speech Recognition}) \cite{Zuluaga-Gomez2023Virtual}.
    \item Un conversor de lenguaje a instrucciones de alto nivel (\textit{Language Parameter and Instruction Parser}).
    \item Un módulo de interfaz de un simulador de vuelo.
    \item Un módulo de respuesta (\textit{Response Handler})
    \item Un módulo de TTS (\textit{Text-to-Speech})
\end{enumerate}

\section{Estudio del \textit{State of the Art}}


\subsection{\textit{Automatic Speech Recognition} (ASR)}

\subsubsection{Desafíos}
\begin{enumerate}
    \item \textbf{Ruido}. Las comunicaciones de tráfico aéreo entre pilotos y controladores utilizan \textbf{amplitud modulada} (AM) \cite{WhatRadioEquipmentPilotsUseToCommunicateWithATC2018}. Esto provoca susceptibilidad a ruido, estática y solapamiento, lo que interfiere con la capacidad de un componente ASR para traducir eficientemente los audios a texto. Los receptores VHF (\textit{Very High Frequency}) son especialmente susceptibles a un nivel demasiado bajo de SNR (\textit{Signal to noise}) \cite{ZuluagaGomez2023Lessons} \cite{ChenKopaldMaTarakanWei2021ATCSpeechRecognition}.
    \item \textbf{Diferencias en el lenguaje, acentos y pronunciaciones} de los controladores. Ocasionalmente, los controladores de distintas regiones prefieren hablar su lengua nativa con pilotos que compartan la región (por ejemplo, pilotos de Iberia pueden hablar en español con controladores de torres de control españolas). La diferencia fonética en acentos (inglés contra indio, por ejemplo) puede suponer un desafío adicional \cite{SimpleFlying_NonEnglishATCPilotCommunicationsGuide} \cite{wee2024adapting}.
    \item \textbf{Fraseología y jerga aeronáutica} que no es convencional respecto al lenguaje hablado diariamente. Términos como \textit{callsign}, \textit{runway}, \textit{climb} pueden no coincidir con el conjunto del lenguaje objetivo de los modelos convencionales \cite{ZuluagaGomez2024ATCO2} \cite{Fan2024CustomizationASR}.
    \item \textbf{Puntos críticos del lenguaje} como los \textit{callsign} o las magnitudes de las instrucciones proporcionadas presentan puntos únicos de fallo. Un fallo en la magnitud de las instrucciones o a quién van dirigidas compromete por completo la maniobra que se quiera comunicar \cite{ZuluagaGomez2021Contextual} \cite{ATCO2_EndToEndCallsignRecognition2021}.
\end{enumerate}

\subsubsection{\textit{Datasets} relevantes}
\begin{enumerate}
    \item \textbf{ATCO2} \cite{ATCO2_EndToEndCallsignRecognition2021}
    \begin{itemize}
        \item Se trata del estándar actual. Son 5.000 horas de audio real de comunicaciones de ATC. Incluye transcripciones y anotaciones.
    \end{itemize}
    \item \textbf{ATCOSIM} \cite{hofbauer-etal-2008-atcosim}
    \begin{itemize}
        \item Modelo útil para validación. 10 horas de grabaciones en inglés, con transcripciones `limpias'.
    \end{itemize}
    \item \textbf{UWB-ATCC} \cite{Jzuluaga_uwb_atcc}
    \begin{itemize}
        \item 20 horas de grabación en inglés. Contiene transcripciones hechas a mano. Otro corpus de datos útil para validación.
    \end{itemize}
    \item \textbf{Malorca ATC Corpus} \cite{malorca_project}
    \begin{itemize}
        \item Aeropuertos de Praga y Viena. 14 horas de grabación. Existen datos de validación (4 horas) y de entrenamiento (10 horas).
    \end{itemize}
\end{enumerate}

\subsubsection{Enfoques actuales y soluciones}
EL \textbf{WER} (Word Error Rate) se define de la manera visible en la ecuación \ref{eq:wer}

\begin{equation}\label{eq:wer}
    WER = \frac{S + D + I}{N}
\end{equation}

donde $S$ son las sustituciones (palabras que se confunden con otras), $D$ son los borrados (\textit{deletions}, palabras que no son detectadas), $I$ son las inserciones (palabras que se añaden) y $N$ es el número total de palabras en la grabación original. \textbf{La mejor solución será aquella que consiga el $WER$ más bajo}.

\begin{enumerate}
    \item \textbf{ATCO2 Corpus} \cite{ATCO2_EndToEndCallsignRecognition2021} No es una solución en sí misma: es un ecosistema estandarizado consistente de, a saber, \textit{ATCO2-test-set corpus} y \textit{ATCO2 pseudo-labeled set corpus}. El primero son cuatro horas de transcripciones comprobadas, marcas con el rol del hablante (a modo \texttt{<pilot>...<pilot>}) además de otros \textit{tags} sobre elementos de la comunicación, además de todo lo que contiene el segundo (salvo la duración). El segundo consiste en aproximadamente 5281 horas de audio de control aéreo, con metadatos como puntuación de detección de inglés, puntuaciones de confianza o SNR; datos contextuales y transcripciones con ASR (pueden no ser correctos).

    Actualmente, una semana de trabajo por transcriptores profesionales (controladores aéreos retirados o en activo) equivale a una hora de transcripciones efectivas para poder ser usadas al estándar \textbf{test-set}. Los datos de diferentes aeropuertos difieren por convenciones locales, lo cual dificulta que haya un dominio estandarizados de datos. A nosotros, sin embargo, no nos afecta para nuestra resolución de problemas, pues buscamos aplicar esto a pseudopilotos y no a control aéreo real.

    Antes del \textit{ATCO2 Corpus}, se intentaba entrenar con \textbf{corpora} como Librispeech \cite{panayotov2015librispeech} o TED-LIUM \cite{hernandez2018ted} pero estos \textit{datasets} probaron no ser lo suficientemente `difíciles' como para entrenar a los ASR de ATC.

    Los resultados del entrenamiento se pueden ver en \cite{ATCO2_EndToEndCallsignRecognition2021}. Encontramos que el WER en ATCO2 test set es en torno a 0,22. 
    
    \textbf{PROPUESTA: }Si bien esto no es aceptable para un entorno de control aéreo real, quizá podamos implementar un mejor método para minimizar el WER. Quizá podamos hacer un nuevo corpus complementario al ATCO2 con datos de entrenamiento. También podemos seguir con un modelo de 0,22 teniendo en cuenta que el audio del pseudopiloto va a tener alta fidelidad y confianza, no necesitamos que el WER sea muy grande (y quizá tampoco necesitemos un corpus de entrenamiento especializado).

    \item \textbf{ATCOSIM Corpus}. Se trata de una base de datos de 10 horas de grabaciones ATC. Todo está en ingles y pronunciado por angloparlantes nativos. En este caso, tampoco se trata de una solución al problema sino de un corpus de entrenamiento, quizá útil para validación de un modelo ASR.

    En este punto conviene preguntarse si no existe un corpus en español que nos permita hacer todo esto pero... en español. \textbf{NO EXISTE} un corpus público que sea ad-hoc para ATC en español. Existen bases de datos de corpus generales (\textbf{Mozilla Common Voice} \cite{ardila2020common}, \textbf{VoxPopuli} \cite{wang-etal-2021-voxpopuli})

    \item \textbf{A Virtual Simulation-Pilot Agent for Training of Air Traffic Controllers} \cite{Zuluaga-Gomez2023Virtual} 
    
    \begin{itemize}
        \item \textbf{APUNTE: } no tanto que ver con ASR, pero se asegura que la pipeline se construye solo con Open Source. 
    \end{itemize}

    \textbf{Wav2Vec2.0} es la herramienta usada para el ASR de este artículo. También usan \textbf{XLSR} (\textit{Cross-Language Speech Recognition}). Merece la pena ampliar un poco esta sección y explicar qué son estas dos cosas.
    \begin{itemize}
        \item \textbf{Wav2Vec2.0 [NO RIGUROSO]} \cite{DBLP:journals/corr/abs-2006-11477}. Para entender esta herramienta explico lo que es el aprendizaje semi supervisado. Supongamos que un fragmento de audio está unívocamente determinado por una serie de informaciones auditivas $a_1, a_2, \ldots, a_n$. Digamos que cada uno de los vectores $a_i$ se compone de $m$ componentes (es decir, $a_i \in \mathbb{R}^m)$. Entonces, lo que hacemos es quitar algunos de los fragmentos de audio y dar al modelo a elegir entre ciertas opciones para rellenar el hueco vacío.
        \item \textbf{XLSR} Es una iniciativa por la que se entrena al modelo anterior en múltiples idiomas. Se utiliza \textit{Multilingual LibriSpeech} \cite{Pratap2020MLSAL} como corpus de entrenamiento.
    \end{itemize}
    El \textbf{resultado} consiste
\end{enumerate}


\subsection{\textit{Language Parameter and Instruction Parser} (LPIP)}

\subsubsection{Desafíos}
\begin{enumerate}
    \item \textbf{Fraseología variable}. Un controlador aéreo puede plantear la misma instrucción a una aeronave de distintas maneras (aunque exista el estándar OACI \cite{eurocontrol_icao_nodate}, de tal forma que muchas instrucciones pueden tener la misma identidad. Por ejemplo:
    \begin{center}
        \texttt{IBERIA292 maintain three five zero zero}
        
        \texttt{IBERIA292 climb to three five zero zero}
    \end{center}
    El parseador óptimo debería interpretar ambas instrucciones como una petición de colocarse a una altura de 3500 pies.
    \item \textbf{Errores de transcripción} del módulo ASR pueden hacer difícil la resiliencia de todo el pseudopiloto. Un caso de ejemplo es el siguiente:
    \begin{center}
        \texttt{IBERIA292 climb to three five oh oh}
    \end{center}
    que quiere decir lo mismo de antes: es una instrucción para colocarse a 3500 pies. Este tipo de fraseología alternativa es crítica pues puede darse en \textit{callsigns} o en valores de instrucción. Un fallo en la interpretación de cualquiera de estos supone un punto único de fallo.
    \item \textbf{Identificación precisa de \textit{callsigns}} \cite{callsign_recognition}, que contienen números y pueden confundirse con magnitudes de instrucciones. Por ejemplo, podríamos pensar en la siguiente transcripción del ASR:
    \begin{center}
        \texttt{IBERIA two nine two zero heading}
    \end{center}
    Un posible parseador de magnitudes podría no saber dónde acaba el \textit{callsign} y dónde empieza la magnitud de la instrucción.
    \item \textbf{Múltiples parámetros en la misma instrucción}, como se ilustra que existen en \cite{Zuluaga‑Gomez2023Virtual}. Podríamos pensar en la siguiente transcripción:
    \begin{center}
        \texttt{climb to FL350, turn left heading 090, speed 250}
    \end{center}
    donde tenemos tres parámetros (altura, rumbo y velocidad) para una sola aeronave. 
    \item \textbf{Estructuras temporales encadenadas}, donde se proporcionan instrucciones secuenciales a aeronaves. Es esencial mantener la cohesión temporal en las instrucciones:
    \begin{center}
        \texttt{climb to FL350, after reaching point X descend to 3000}
    \end{center}
    \item \textbf{Normalización}, de las unidades y parámetros de instrucciones. \texttt{FL350}, \texttt{Flight Level three five zero}, \texttt{three five zero} son maneras de decir 35000 pies, y deben interpretarse como tal.
    \item \textbf{Resiliencia ante espontaneidad del discurso}, ya que instrucciones que involucran correcciones de ellas mismas deben ser interpretadas como si fueran instrucciones normales.
\end{enumerate}

\subsubsection{Enfoques actuales y soluciones}
\begin{enumerate}
\item \textbf{A Virtual Simulation-Pilot Agent for Training of Air Traffic Controllers} \cite{Zuluaga-Gomez2023Virtual} 

Se define un componente de \textit{high level parsing} y se le llama \textit{High Level Entity Parser} (HLEP). Cada una de las instrucciones transcritas se utiliza como entrada a este módulo, donde se extraen los campos comunes a todas las instrucciones ATC. Cada una de las entidades que se pueden extraer (en forma de valores etiquetados) se llama \textit{Named entity}. El reconocimiento de las \textit{Named entities} recibe el nombre de \textit{Named Entity Recognition}:

\begin{itemize}
    \item \textit{Callsign} (por ejemplo \texttt{Lima Echo Sierra 3 3 5} $\rightarrow$ \texttt{LES-335})
    \item Comando del ATCO (mantener una altitud, orientarse con determinado ángulo...)
    \item Valores de la instrucción
\end{itemize}

De esta manera, una transcripción del tipo

\texttt{ryanair nine two bravo quebec turn right heading zero nine zero}

se transforma en una instrucción etiquetada, de la forma

\texttt{<callsign>}\textcolor{blue}{\texttt{ryanair nine two bravo quebec}}\texttt{</callsign>} 

\texttt{<command>}\textcolor{red}{\texttt{turn right heading}}\texttt{</command>}

\texttt{<value>}\textcolor{green}{\texttt{zero nine zero}}\texttt{</value>}

Para esta tarea se emplea un Modelo de Lenguaje pre-entrenado y se sigue la estrategia de ajuste fino para la NER. El modelo utilizado fue BERT (\textit{Bidirectional Encoder Representations From Transformers}) (específicamente la versión pre-enetrenada \textbf{BERT-base-uncased}).

Se ajustó a la tarea utilizando el corpus ATCO2 \cite{ATCO2_EndToEndCallsignRecognition2021}. Este \textit{dataset} es altamente conveniente para el \textit{fine-tuning} pues cuenta con transcripciones ya etiquetadas.

En términos de los resultados, el sistema basado en \textbf{BERT} consiguió un resultado de precisión del \textbf{97,5\%} en detección de \textit{callsigns}, mientras que consiguió una del \textbf{82\%} en \textit{commands} y del \textbf{87,2\%} en \textit{values}.

\item \textbf{Research on the Method of Air Traffic Control Instruction Keyword Extraction Based on the Roberta-Attention-BiLSTM-CRF Model} \cite{aerospace12050376}



\end{enumerate}

\section{FlightGear, simulador de vuelo}

\newpage
\clearpage
\printbibliography
\end{document}
