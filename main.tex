\documentclass[12pt, a4paper]{book}

% --- PAQUETES BÁSICOS ---
\usepackage[utf8]{inputenc}      % Codificación de entrada UTF-8
\usepackage[T1]{fontenc}         % Codificación de fuentes para caracteres especiales
\usepackage[english]{babel}      % Soporte para el idioma español (acentos, guiones, etc.)

\usepackage{ebgaramond}          % Fuente principal del texto (Serif, Garamond)
\usepackage[scaled=0.9]{helvet}  % Fuente para títulos (Sans-Serif, Helvetica)

% --- DISEÑO DE PÁGINA ---
\usepackage[margin=2.5cm]{geometry} % Márgenes generosos
\usepackage{setspace}               % Para controlar el interlineado
\usepackage{svg}
\svgsetup{inkscapeexe={/Applications/Inkscape.app/Contents/MacOS/inkscape}}
\onehalfspacing                     % Interlineado de 1.5 (puedes usar \setstretch{1.2})

\usepackage{titlesec}

\usepackage{hyperref}            % Para crear hipervínculos (en PDF)

% Formato para \chapter
% Estilo "display": el título y el número van en párrafos separados
\titleformat{\chapter}[display]
  {\normalfont\sffamily\huge\bfseries\centering} % Formato para todo el bloque: Sans-Serif, huge, negrita, centrado
  {Chapter \thechapter}                          % El texto "Chapter" + número
  {20pt}                                         % Espacio vertical entre el número y el título
  {\Huge}                                        % Formato solo para el nombre del capítulo (ej. "Introduction")

% Formato para \section
\titleformat{\section}
  {\normalfont\sffamily\Large\bfseries} % Formato: Sans-Serif, Large, negrita
  {\thesection}                         % El número de la sección
  {1em}                                 % Espacio horizontal entre el número y el título
  {}                                    % Código extra (no necesario aquí)


\usepackage{csquotes}
\usepackage{booktabs}
\usepackage[backend=biber]{biblatex}
\usepackage{graphicx} % Required for inserting images
\usepackage{geometry}
\usepackage{amsmath}
\usepackage{amssymb}
\usepackage{booktabs}
\usepackage{tikz}
\usetikzlibrary{positioning}
\usepackage{xcolor}
\usepackage{array}
\usepackage{longtable}
% beta 
% \pagecolor[rgb]{0,0,0} \color[rgb]{1,1,1}
\DeclareUnicodeCharacter{202F}{\,}
\addbibresource{references.bib}

% URL breaking configuration (biblatex already loads url package)
\def\UrlBreaks{\do\/\do-\do_\do.\do=\do?\do&}
\urlstyle{same}


% Better line breaking to handle overfull hbox
\sloppy

\begin{document}

\begin{titlepage}
	\centering % Center all content on the page

	% --- University and School Name ---
	{\Large UNIVERSIDAD AUTÓNOMA DE MADRID}
	\par
	{\Large ESCUELA POLITÉCNICA SUPERIOR}

	\vfill

	\includegraphics[width=0.35\textwidth]{assets/logo_eps.png}
	\hspace{1.5cm}
	\includegraphics[width=0.35\textwidth]{assets/uam_logo.png}

	\vfill % Adds flexible space

	{\Large Grado en}
	\par
	{\Large Ingeniería Informática}

	\vspace{2cm}

	% --- "MASTER THESIS" ---
	{\Huge\textbf{TRABAJO DE FIN DE GRADO}}

	\vspace{2cm} % Adds a fixed vertical space

	\begin{center}
		{\Huge \textbf{Extending Virtual Pseudo-Pilot Automation with Flight Simulation Interface Integration}}
	\end{center}

	\vfill % Adds flexible space

	% --- Author and Advisor ---
	{\LARGE Ignacio Ildefonso de Miguel Ruano}
	\par
	\vspace{0.5cm} % A little space between author and advisor
	{\LARGE Advisor: Eduardo Cermeño Mediavilla}

	\vfill % Adds flexible space

	% --- Date ---
	{\LARGE DATE}

\end{titlepage}

\newpage

\renewcommand{\normalsize}{\fontsize{14pt}{17pt}\selectfont}
\normalsize

\chapter{Introduction}
\label{chap:intro}
The increasing growth of the world of aviation has begun to raise issues regard


\chapter{Related Work}
\label{chap:related-work}
\section{Proposed system architecture}
Based on our previous analysis of the State of the Art, we can now move on to our proposal, which \textbf{aims to fully close the gap between the partially automated system and a fully automated one}

Up until now, a pseudopilot is required in order to transform the trainee ATC's instructions into real, inputtable information for a flight simulation engine. We propose an architecture consisting of two main components.

As established in Chapter~\ref{chap:related-work}, the lack of automated execution is the only aspect separating current implementations of the full automation. We address this by proposing an architecture like the one portrayed in Figure~\ref{fig:full-auto-pseudopilot}

\begin{figure}[htbp]
	\centering
	\includegraphics[width=1\textwidth]{assets/diagram_proposal/proposal.png}
	\caption{Fully automated pseudo-piloting system. The Trainee ATC first inputs some instructions via voice, which then get sent to an ASR module to transform said instructions into text. The output of this transformation is then fed into a Language Parameter and Instruction Parser module, which reads from a context database in order to improve parsing accuracy.
		The parsed instructions then get fed into a Pseudopilot Instruction Execution Module, which communicates with a Flight Simulation Engine. The feedback from said engine is then converted into a response via the Response Builder module, which is then transformed into a speech output via a TTS module. This speech output is presented to the Trainee ATC, finalizing the loop.}
	\label{fig:full-auto-pseudopilot}
\end{figure}

This system is designed as a sequential pipeline of events, starting by the instructions emitted by the ATC, and ending in feedback presented to the same individual. The system is composed by the following key modules:

\begin{enumerate}
	\item \textbf{Automatic Speech Recoginition (ASR)} or \textbf{Voice Recognition}: the pseudo-pilot's ears. This module's key and only role is to transform voice input into raw text, which will be parsed by the LPIP.
	\item \textbf{(Optional) Context Database}: allows the pseudo-pilot to remember the current state of the aircrafts: their callsigns, the last instructions given to them, etc.
	\item \textbf{Language Parameter and Instruction Parser (LPIP)}: this module extracts the main elements necessary to execute an instruction. These are, namely, the \textbf{callsign}, \textbf{instruction} and \textbf{parameters/values}. It can also read from the \textbf{Context Database} to resolve faulty voice recognition outputs.
	\item \textbf{Pseudopilot Instruction Execution Module (PIEM)}: serves as the interface that connects the instructions of the ATC to the flight simulation engine. It reads feedback from this same engine and can use it to update the \textbf{Context Database} if present.
	\item \textbf{Flight Simulation Engine}: receives instructions from the \textbf{PIEM} and sends back feedback on the affected aircraft.
	\item \textbf{Response Builder}: processes feedback received by the \textbf{PIEM} and converts it into raw text containing the response addressed to the ATC.
	\item \textbf{Text-to-speech}: transforms the response's raw text into a speech response, which will be heard by the \textbf{ATC}.
\end{enumerate}

\section{Component Selection}
In order to guarantee that the system works as intended, it is crucial to choose optimal components for each of the steps of our pipeline described in the previous section. For this, we have done extensive research on methods implemented for similar purposes of each step.

\subsection{ASR Module}
As we have already specified, the ASR serves as the pseudo-pilot's `ears'. Its performance is critical for the entire pipeline. If not handled correctly, this could constitute a single point of failure for the whole pseudopilot architecture. Significant challenges are present when we consider the domain of ATC problems.

Current research in the `ASR for ATC' field takes into account several challenges such as \textbf{noise}, since traditional ATC communications run on modulated amplitude~\cite{WhatRadioEquipmentPilotsUseToCommunicateWithATC2018}. So-known `Very High Frequency' receptors are very susceptible to a low Signal-to-Noise ratio \cite{ZuluagaGomez2023Lessons} \cite{ChenKopaldMaTarakanWei2021ATCSpeechRecognition}, present in ATC communications.
Luckily for us, the problem we tackle need not have this constraint, as we would be running a simulated environment where there's no need to use modulated amplitude.
Other challenges in the area of \textbf{language differences, as well as accents and pronunciations} are also negligible for our current purposes, as we will consider the pseudo-pilot handles all communications in \textbf{english}. It is important to note, however, that these challenges do exist \cite{SimpleFlying_NonEnglishATCPilotCommunicationsGuide} \cite{wee2024adapting}.

Other conventional challenges do suppose crucial points to consider. Current ASR models have a wide variety of usages, and \textbf{aeronautical phraseology} could be, in some cases, too convoluted for these models. Terms like `callsign', or `runway' could not coincide with the target language registry of conventional models \cite{ZuluagaGomez2024ATCO2} \cite{Fan2024CustomizationASR}.

Finally, we must consider this model's lack of resilience: a failure in any of the transcript supposes wrong instructions being fed into the pseudo-pilot, and thus not being useful for the ATC training \cite{ZuluagaGomez2021Contextual} \cite{ATCO2_EndToEndCallsignRecognition2021}.

This leaves us two main problems to solve, or rather, two aspects we would like to have in our ASR module. Namely, we want it to \textbf{minimize errors per word} and be \textbf{resilient, resisting to transcription mistakes}.

For the first problem, the metric of \textbf{WER} (Word Error Rate) is useful. It's defined as seen in Equation~\ref{eq:wer}

\begin{equation}\label{eq:wer}
	WER = \frac{S + D + I}{N}
\end{equation}

Where $S$ are substitutions (words confused with others), $D$ are deletions (words not detected at all), $I$ are insertions (non-existent words inserted into the transcription) and $N$ is the total number of words in the original recording.

In order to obtain models with a low WER, researchers use the \textbf{ATCO2 Corpus} \cite{ATCO2_EndToEndCallsignRecognition2021} as the most relevant dataset. It's a standardized ecosystem, consistent of, namely, the \textit{ATCO2-test-set corpus} (four hours of checked transcriptions, marks with speaker roles [\texttt{<pilot>...<pilot>}, etc.] as well as other tags that mark communication elements\ldots) and the \textit{ATCO2 pseudo-labeled set corpus} (approximately 5281 hours of ATC audio, with metadata like english detection scores, trust scores or SNR; contextual data and ASR-made transcriptions which may or may not be correct).
Researchers also mention \textbf{ATCOSIM Corpus} \cite{hofbauer-etal-2008-atcosim}, which consists of 10 hours of ATC recordings, in english and pronounced by native english speakers.

Other solutions like Librispeech \cite{panayotov2015librispeech} or TED-LIUM \cite{hernandez2018ted} proved to not be `challenging' enough to train ATC-domain ASR. Some sample results of training can be seen in \cite{ATCO2_EndToEndCallsignRecognition2021}, which are around 0.22 (not considering our negligible environment).

Research in \cite{Zuluaga-Gomez2023Virtual} attempts to tackle this problem using Open Source tools. Zuluaga et. al propose the use of \textbf{Wav2Vec2.0}, as well as \textbf{XLSR} (\textit{Cross-Language Speech Recognition}). As we already discussed, XLSR doesn't seem reasonable for our current problem setting, and thus the replication of this paper doesn't seem entirely fitting for our purposes.

Other research such as~\cite{vanDoorn2023} proposes fine-tuning OpenAI's Whisper \cite{OpenAI_Whisper_2022} using the formerly mentioned \textbf{ATCO2} and \textbf{ATCOSIM} datasets. This research is fully reproducible by us, (except maybe for the fine-tuning, which requires high level hardware) since the method is described in detail in the paper. With methods of normalization and prompting, the ASR shows promising results of around a 0.1 WER for some datasets.

In contrast, the problem of \textbf{resilience} cannot be solved by the ASR model itself, as there's no way the ASR model could "monitor itself", and as such will be solved later with other component (\textbf{Context Database}).

Taking all of this into account, we have decided to use Whisper Large v2, trained on ATCO2 and ATCOSIM, with an expected WER of around $0.12$.

\subsection{Language Parameter and Instruction Parser}
This step is standard in other research papers, and thus challenges for its correct functioning are well defined. The main focus centers around problems like \textbf{variable phraseology}, where an ATCo could
state the same instruction to an aircraft in different ways, despite the ICAO standard \cite{eurocontrol_icao_nodate}. Various instructions could then have the same intent:

\begin{center}
	\texttt{IBERIA292 maintain three five zero zero}

	\texttt{IBERIA292 climb to three five zero zero}
\end{center}

An optimal \textbf{LPIP} should identify both of these as `Flight with callsign IBERIA292 should maintain an altitude of 3500 feet'.
Other limitations, albeit more obvious, include errors in the \textbf{ASR} step of the pipeline, which might end up producing transcripts that do not align with the original intent of the message (this is where our \textbf{Context Database} will come in handy later).
Take the following transcription:

\begin{center}
	\texttt{IBERIA292 climb to three five oh oh}
\end{center}

This alternative formulation could show up in callsigns or in values. Failing to interpret these correctly means compromising the whole pipeline.

Additionally, callsigns contain information which could (from a parsing perspective), be confused with command values. Of course, this also adds a single point of failure, as failing to get the callsign correctly makes the instruction assignment be faulty.
This could be a problem with transcriptions like:
\begin{center}
	\texttt{Iberia two nine two zero heading}
\end{center}

Without a context, it's difficult to know where the callsign ends and the values for the instruction begin. Other problems include multiple parameters in the same instruction:
\begin{center}
	\texttt{climb to FL350, turn left heading 090, speed 250}
\end{center}
Making it impossible for traditional parsers to correctly partitionate the message with one concrete command per each instruction. While we might experience multiple commands per instruction,
we also might experience multiples commands per instruction followed by a temporal structure of said commands. It is also crucial to represent these correctly:
\begin{center}
	\texttt{climb to FL350, after reaching point X descend to 3000}
\end{center}

A normalization of the units and instruction parameters is also necessary. \texttt{FL350}, \texttt{flight Level three five zero} or simply \texttt{three five zero} are distinct yet equivalent
ways of saying `35,000 feet'. Spontaneity and self-correction within instructions is to be expected as well, as one instruction may contain elements which are hierarchically superior to others.
\begin{center}
	\texttt{Iberia two nine turn leaft heading 045, umm, sorry, 090}
\end{center}

While these problems do pose major challenges, current advances in research have made an effort to mitigate them. In \cite{Zuluaga-Gomez2023Virtual}, researchers present a `High Level Entity Parser' module.
Each and every one of the transcribed instructions are passed as input to this parser, where fields present in every ATC instruction are extracted. Each of these fields is called a `Named entity'. The recognition
of said entities is called `Named Entity Recognition'. Each named entity can fall in one of three categories: `callsign' or identifier of an aircraft; `command', which is the basic verb associated to the instruction and `value',
which indicates the magnitude of the extent to which the command is to be executed.

This way, a transcription like
\begin{center}
	\texttt{ryanair nine two bravo quebec turn right heading zero nine zero}
\end{center}
Is parsed and transformed with this NLP, identifying its named entities, with its output being \texttt{<callsign>\textcolor{blue}{ryanair nine two bravo quebec}</callsign>
	<command>\textcolor{red}{turn right heading}</command><value>\textcolor{green}{zero nine zero}</value>}

For this task, a pre-trained language model is used, fine-tuning it for this NER task. The model used was BERT~\cite{devlin2019bertpretrainingdeepbidirectional}. The model was tuned
using the previously described ATCO2 corpus \cite{ATCO2_EndToEndCallsignRecognition2021}. This dataset is highly convenient for the Named Entity Recognition task, as data is already presented
in a named entity format. Result-wise, BERT achieved a precision of 97.5\% in callsign detection, while achieving 82\% in commands and 87.2\% in values.

This method, however, failed to capture contextual dependencies within segments of an ATC communication. There was a difficulty to maintain the correct sequence for keywords. There was an attempt to
solve this by \cite{aerospace12050376}, proposing a new model: `Roberta-Attention-BiLSTM-CRF'. Previous models could only capture dependencies within the same instruction or instruction segment. This new model
can capture semantic relevance throughout several instruction segments.

An attention module is used, allowing them to capture the relation between sequential instructions. The `BiLSTM' stands for Bidirectional Long Short-Term Memory, which is used to capture contextual dependencies of used terms.

The CRF layer (Conditional Random Field) attempts to predict keywords in the correct sequential order. It establishes sequential dependencies between multiple keywords. Similarly to \cite{Zuluaga-Gomez2023Virtual}, this approach
parses each ATC instruction in named entities such as `callsigns', `commands' and `values'. A new named entity is added, the `area code', which represents the airspace or control area. Consider the transcription

\begin{center}
	\texttt{UAL215, Barajas Tower, cleared for take off, then contact departures}
\end{center}

Without CRF or Attention Module, BERT could interpret the remaining text after `Barajas Tower' as one single instruction. Even if BERT was to correctly identify both instructions as separate ones,
there's still the matter of sequentiality, which BERT could mistakenly rearrange. Without contextual bidirectionality, we are also not guaranteed to maintain the callsign to which a controller refers.

In terms of results, Roberta-Attention-BiLSTM-CRF surpasses any other pre-existing model with similar configurations. In terms of precision of named entity extraction, the accuracy is approximately of \textbf{0.9}.
The individual effectiveness of each component is evidenced by the increase of this accuracy score with the addition of said individual components in each iteration. Table~\ref{tab:resultados-modelos} showcases this
phenomenon.

\begin{table}[ht]
	\centering
	\caption{Model result}
	\label{tab:resultados-modelos}
	\begin{tabular}{lcccc}
		\toprule
		\textbf{Model}               & \textbf{Acc} \\
		\midrule
		BiLSTM-CRF                   & 0.805        \\
		Roberta-BiLSTM               & 0.832        \\
		Roberta-Attention-BiLSTM     & 0.808        \\
		Roberta-BiLSTM-CRF           & 0.869        \\
		BERT-BiLSTM-CRF              & 0.865        \\
		Roberta -LSTM-CRF            & 0.855        \\
		BERT-Attention-BiLSTM-CRF    & 0.886        \\
		Roberta-Attention-BiLSTM-CRF & 0.895        \\
		\bottomrule
	\end{tabular}
\end{table}

The research showcased in \cite{doi:10.2514/6.2024-4359} emerges from the necessity to decongest the ATC radio communications. Nonetheless, it is aplicable to our situation.
Researchers use an \textit{Intent Classification} + \textit{Slot Filling} system. The first one gives us what we have previously called
`Named Entity Recognition for Commands' while the latter gives the `Named Entity Recognition for Callsign and Values'.

More recent research in the field of Small Language Models~\cite{belcak2025smalllanguagemodelsfuture}~\cite{AguileraEnhancingAviation2024}
suggests High Level Language Parsing could be better performed by a model which acts as a `black box' with context and instructions.

The problem we face does not have the disadvantage of unclear diction. Communication channels are to be controlled, as the ATC environment is not a real one.
Taking into account all previous research, it seems reasonable to suppose we need to parse the three main elements: `callsign', `command', `values'~\cite{Zuluaga-Gomez2023Virtual}~\cite{doi:10.2514/6.2024-4359} from the ASR output.
We also need a deeper understanding of the bidirectional context rather than just a parser, as evidenced by \cite{aerospace12050376}.

With all of this in mind and given that the scope of this research, it seems reasonable that choosing a \textbf{Small Language Model}, is the most appropriate
of solutions. Specifically, a model like \textbf{GPT-4o-mini}, with an OpenAI API. The choice of this model is justified with its dominion in the benchmarks~\cite{OpenAIGPT4omini2024}.
This solution also allows us to implement an all-in-one approach, where this SLM would also generate response candidates and pass them to the response generator. This solution allows us to leverage **structured outputs (JSON)** natively, eliminating the need for complex regex post-processing required by NER models. Furthermore, its 'zero-shot' reasoning capabilities provide the resilience needed to handle disfluencies and self-corrections without specific fine-tuning.

The traditional architecture would result in a similar schema to the one visible in Figure~\ref{fig:non-naive-solution}, while the one we propose would result in a simpler operation, like the one described
in Figure~\ref{fig:naive-solution}

\begin{figure}[h]
	\centering
	\includegraphics[width=\linewidth]{assets/non-naive-approach/non-naive-approach.png}
	\caption{Method discussed in recent literature for language parsing, described in a diagram}
	\label{fig:non-naive-solution}
\end{figure}

\begin{figure}[h]
	\centering
	\includegraphics[width=\linewidth]{assets/naive-approach/naive-approach.png}
	\caption{Method using an LPIP module whose main component is GPT-4o-mini (or any other Small Language Model)}
	\label{fig:naive-solution}
\end{figure}

As seen in Figure~\ref{fig:naive-solution}, we are finally ready to talk about the \textbf{Context Database}. To finalize this section, we
present the prompt with which we feed the Small Language Model. This prompt must make the language model transform the input into a \texttt{json} object with the format in
Listing~\ref{lst:system-prompt}

\begin{lstlisting}[language=json, caption={System Prompt for LPIP}, label={lst:system-prompt}]
You are an ATC instruction parser.
Your task is to extract:
1. Callsign (normalized).
2. Command (CLIMB, TURN, CONTACT, etc.).
3. Value (normalized to integer).

You must also construct reasonable response messages (readbacks), where you have creative freedom:
1. success_msg: The standard readback if the instruction is understood.
2. error_msg: The response if the instruction is invalid or ambiguous.

Output must be strictly JSON format:
{
	"callsign": "IBE292",
	"command": "CLIMB",
	"value": 35000,
	"success_msg": "Roger that, IBE292 climbing to 35000 feet",
	"error_msg": "Station calling, say again, instruction unclear"
}
\end{lstlisting}

\subsection{Context database}
As we have stated in both the \textbf{LPIP} and \textbf{ASR} section, it is crucial that our instruction building process (first being correctly recognized, then being correctly built) is resilient.
In case of a wrong transcription, an ideal scenario would involve our LPIP (Small Language Model) being able to correct the transcription using its judgement. For this, we introduce the Context Database.

This component functions as the LPIP's memory. In the domain of Language Models, this technique is a simplified form of \textbf{Retrieval-augmented generation}~\cite{rag}. It will provide our LPIP with contextual data preventing it from hallucinating and adding a resilience layer.

The absence of this component would mean the LPIP operates in the dark, with no information of the simulation state at the time of generating the instructions or the response. Without contextual data, instructions
without a callsign would not have any aircraft to be directed to, as the LPIP would not be able to infer from just the transcription the last aircraft to which an instruction was directed. This is only one amongst
the many problems the Context Database helps solve.

Another scenario is one where the ASR module has mistakenly transcribed an audio fragment. This could look something like \texttt{IBERIA two nine true (\ldots)}. Without a contextual database, the LPIP
could mistakenly assume the callsign to be something erroneous. However, given that the Contextual Database injects context data containing the callsigns of the available aircrafts (\texttt{"IBERIA292", "LUXAIR331"}), the SLM
will perform a \textbf{phonetic match} between its transcription and the only possible data.

\subsubsection{Implementation, data structures, update loop and final prompt}
Using an actual database, like Redis or an SQL-based database, seems like overkill. We need a lightweight agent which operates with fast updates and fast reads. We don't need it to be persistent, so we will be using a Python dictionary database.
Each of its elements will be a serializable object.

This database will hold \textbf{Aircraft} elements. The key will be their callsign. Each of these aircrafts will contain certain attributes:

\begin{enumerate}
	\item \textbf{Current state:} Aircraft's altitude, heading, speed, latitude and longitude
	\item \textbf{Last known instruction:} Last instruction to have been sent to this aircraft.
	\item \textbf{Last instruction was successful:} Either true or false.
\end{enumerate}

This contextual database is dynamic, and updated as described in Figure~\ref{fig:full-auto-pseudopilot}. The \textbf{PIEM} (Pseudo-pilot Instruction Execution Module) is responsible for updating this database every time the Flight Simulation interface
provides feedback on the instructions.

This architecture ensures the LPIP always has access to the most up-to-date information, allowing it to further improve its resilient architecture. All in all, the prompt directed to the Small Language Model ends up being something like~\ref{lst:system-prompt-updated} (complementing \ref{lst:system-prompt})

\begin{lstlisting}[language=json, caption={System Prompt for LPIP, updated with }, label={lst:system-prompt-updated}]
    [INITIAL INSTRUCTIONS]

    You have been provided an output which may contain mistaken transcriptions. When possible, match the transcriptions to the contextual data, to make sure they make sense:
    [
        "IBE292": {
            "status": [alt, head, speed, lat, long],
            "last_instruction": "CLIMB 35000",
            "last_instruction_successful": True,
        }
    ]
    
\end{lstlisting}
(ADD IMPLEMENTATION TO APPENDIX)


\section{Planteamiento inicial}
El modelo que proponemos consiste en una \textit{pipeline} de cuatro elementos.
\begin{enumerate}
	\item El elemento de \textbf{captura y transcripción de voz}. En este entorno, tradicionalmente se le conoce como \textbf{ASR} (\textit{Automatic Speech Recognition}) \cite{Zuluaga-Gomez2023Virtual}.
	\item Un conversor de lenguaje a instrucciones de alto nivel (\textit{Language Parameter and Instruction Parser}).
	\item Un módulo de interfaz de un simulador de vuelo.
	\item Un módulo de respuesta (\textit{Response Handler})
	\item Un módulo de TTS (\textit{Text-to-Speech})
\end{enumerate}

\section{Estudio del \textit{State of the Art}}


\subsection{\textit{Automatic Speech Recognition} (ASR)}

\subsubsection{Desafíos}
\begin{enumerate}
    \item \textbf{Ruido}. Las comunicaciones de tráfico aéreo entre pilotos y controladores utilizan \textbf{amplitud modulada} (AM) \cite{WhatRadioEquipmentPilotsUseToCommunicateWithATC2018}. Esto provoca susceptibilidad a ruido, estática y solapamiento, lo que interfiere con la capacidad de un componente ASR para traducir eficientemente los audios a texto. Los receptores VHF (\textit{Very High Frequency}) son especialmente susceptibles a un nivel demasiado bajo de SNR (\textit{Signal to noise}) \cite{ZuluagaGomez2023Lessons} \cite{ChenKopaldMaTarakanWei2021ATCSpeechRecognition}.
    \item \textbf{Diferencias en el lenguaje, acentos y pronunciaciones} de los controladores. Ocasionalmente, los controladores de distintas regiones prefieren hablar su lengua nativa con pilotos que compartan la región (por ejemplo, pilotos de Iberia pueden hablar en español con controladores de torres de control españolas). La diferencia fonética en acentos (inglés contra indio, por ejemplo) puede suponer un desafío adicional \cite{SimpleFlying_NonEnglishATCPilotCommunicationsGuide} \cite{wee2024adapting}.
    \item \textbf{Fraseología y jerga aeronáutica} que no es convencional respecto al lenguaje hablado diariamente. Términos como \textit{callsign}, \textit{runway}, \textit{climb} pueden no coincidir con el conjunto del lenguaje objetivo de los modelos convencionales \cite{ZuluagaGomez2024ATCO2} \cite{Fan2024CustomizationASR}.
    \item \textbf{Puntos críticos del lenguaje} como los \textit{callsign} o las magnitudes de las instrucciones proporcionadas presentan puntos únicos de fallo. Un fallo en la magnitud de las instrucciones o a quién van dirigidas compromete por completo la maniobra que se quiera comunicar \cite{ZuluagaGomez2021Contextual} \cite{ATCO2_EndToEndCallsignRecognition2021}.
\end{enumerate}

\subsubsection{\textit{Datasets} relevantes}
\begin{enumerate}
    \item \textbf{ATCO2} \cite{ATCO2_EndToEndCallsignRecognition2021}
    \begin{itemize}
        \item Se trata del estándar actual. Son 5.000 horas de audio real de comunicaciones de ATC. Incluye transcripciones y anotaciones.
    \end{itemize}
    \item \textbf{ATCOSIM} \cite{hofbauer-etal-2008-atcosim}
    \begin{itemize}
        \item Modelo útil para validación. 10 horas de grabaciones en inglés, con transcripciones `limpias'.
    \end{itemize}
    \item \textbf{UWB-ATCC} \cite{Jzuluaga_uwb_atcc}
    \begin{itemize}
        \item 20 horas de grabación en inglés. Contiene transcripciones hechas a mano. Otro corpus de datos útil para validación.
    \end{itemize}
    \item \textbf{Malorca ATC Corpus} \cite{malorca_project}
    \begin{itemize}
        \item Aeropuertos de Praga y Viena. 14 horas de grabación. Existen datos de validación (4 horas) y de entrenamiento (10 horas).
    \end{itemize}
\end{enumerate}

\subsubsection{Enfoques actuales y soluciones}
EL \textbf{WER} (Word Error Rate) se define de la manera visible en la ecuación \ref{eq:wer}

\begin{equation}\label{eq:wer}
    WER = \frac{S + D + I}{N}
\end{equation}

donde $S$ son las sustituciones (palabras que se confunden con otras), $D$ son los borrados (\textit{deletions}, palabras que no son detectadas), $I$ son las inserciones (palabras que se añaden) y $N$ es el número total de palabras en la grabación original. \textbf{La mejor solución será aquella que consiga el $WER$ más bajo}.

\begin{enumerate}
    \item \textbf{ATCO2 Corpus} \cite{ATCO2_EndToEndCallsignRecognition2021} No es una solución en sí misma: es un ecosistema estandarizado consistente de, a saber, \textit{ATCO2-test-set corpus} y \textit{ATCO2 pseudo-labeled set corpus}. El primero son cuatro horas de transcripciones comprobadas, marcas con el rol del hablante (a modo \texttt{<pilot>...<pilot>}) además de otros \textit{tags} sobre elementos de la comunicación, además de todo lo que contiene el segundo (salvo la duración). El segundo consiste en aproximadamente 5281 horas de audio de control aéreo, con metadatos como puntuación de detección de inglés, puntuaciones de confianza o SNR; datos contextuales y transcripciones con ASR (pueden no ser correctos).

    Actualmente, una semana de trabajo por transcriptores profesionales (controladores aéreos retirados o en activo) equivale a una hora de transcripciones efectivas para poder ser usadas al estándar \textbf{test-set}. Los datos de diferentes aeropuertos difieren por convenciones locales, lo cual dificulta que haya un dominio estandarizados de datos. A nosotros, sin embargo, no nos afecta para nuestra resolución de problemas, pues buscamos aplicar esto a pseudopilotos y no a control aéreo real.

    Antes del \textit{ATCO2 Corpus}, se intentaba entrenar con \textbf{corpora} como Librispeech \cite{panayotov2015librispeech} o TED-LIUM \cite{hernandez2018ted} pero estos \textit{datasets} probaron no ser lo suficientemente `difíciles' como para entrenar a los ASR de ATC.

    Los resultados del entrenamiento se pueden ver en \cite{ATCO2_EndToEndCallsignRecognition2021}. Encontramos que el WER en ATCO2 test set es en torno a 0,22. 
    
    \textbf{PROPUESTA: }Si bien esto no es aceptable para un entorno de control aéreo real, quizá podamos implementar un mejor método para minimizar el WER. Quizá podamos hacer un nuevo corpus complementario al ATCO2 con datos de entrenamiento. También podemos seguir con un modelo de 0,22 teniendo en cuenta que el audio del pseudopiloto va a tener alta fidelidad y confianza, no necesitamos que el WER sea muy grande (y quizá tampoco necesitemos un corpus de entrenamiento especializado).

    \item \textbf{ATCOSIM Corpus}. Se trata de una base de datos de 10 horas de grabaciones ATC. Todo está en ingles y pronunciado por angloparlantes nativos. En este caso, tampoco se trata de una solución al problema sino de un corpus de entrenamiento, quizá útil para validación de un modelo ASR.

    En este punto conviene preguntarse si no existe un corpus en español que nos permita hacer todo esto pero... en español. \textbf{NO EXISTE} un corpus público que sea ad-hoc para ATC en español. Existen bases de datos de corpus generales (\textbf{Mozilla Common Voice} \cite{ardila2020common}, \textbf{VoxPopuli} \cite{wang-etal-2021-voxpopuli})

    \item \textbf{A Virtual Simulation-Pilot Agent for Training of Air Traffic Controllers} \cite{Zuluaga-Gomez2023Virtual} 
    
    \begin{itemize}
        \item \textbf{APUNTE: } no tanto que ver con ASR, pero se asegura que la pipeline se construye solo con Open Source. 
    \end{itemize}

    \textbf{Wav2Vec2.0} es la herramienta usada para el ASR de este artículo. También usan \textbf{XLSR} (\textit{Cross-Language Speech Recognition}). Merece la pena ampliar un poco esta sección y explicar qué son estas dos cosas.
    \begin{itemize}
        \item \textbf{Wav2Vec2.0 [NO RIGUROSO]} \cite{DBLP:journals/corr/abs-2006-11477}. Para entender esta herramienta explico lo que es el aprendizaje semi supervisado. Supongamos que un fragmento de audio está unívocamente determinado por una serie de informaciones auditivas $a_1, a_2, \ldots, a_n$. Digamos que cada uno de los vectores $a_i$ se compone de $m$ componentes (es decir, $a_i \in \mathbb{R}^m)$. Entonces, lo que hacemos es quitar algunos de los fragmentos de audio y dar al modelo a elegir entre ciertas opciones para rellenar el hueco vacío.
        \item \textbf{XLSR} Es una iniciativa por la que se entrena al modelo anterior en múltiples idiomas. Se utiliza \textit{Multilingual LibriSpeech} \cite{Pratap2020MLSAL} como corpus de entrenamiento.
    \end{itemize}
    El \textbf{resultado} consiste
\end{enumerate}


\subsection{\textit{Language Parameter and Instruction Parser} (LPIP)}

\subsubsection{Desafíos}
\begin{enumerate}
    \item \textbf{Fraseología variable}. Un controlador aéreo puede plantear la misma instrucción a una aeronave de distintas maneras (aunque exista el estándar OACI \cite{eurocontrol_icao_nodate}, de tal forma que muchas instrucciones pueden tener la misma identidad. Por ejemplo:
    \begin{center}
        \texttt{IBERIA292 maintain three five zero zero}
        
        \texttt{IBERIA292 climb to three five zero zero}
    \end{center}
    El parseador óptimo debería interpretar ambas instrucciones como una petición de colocarse a una altura de 3500 pies.
    \item \textbf{Errores de transcripción} del módulo ASR pueden hacer difícil la resiliencia de todo el pseudopiloto. Un caso de ejemplo es el siguiente:
    \begin{center}
        \texttt{IBERIA292 climb to three five oh oh}
    \end{center}
    que quiere decir lo mismo de antes: es una instrucción para colocarse a 3500 pies. Este tipo de fraseología alternativa es crítica pues puede darse en \textit{callsigns} o en valores de instrucción. Un fallo en la interpretación de cualquiera de estos supone un punto único de fallo.
    \item \textbf{Identificación precisa de \textit{callsigns}} \cite{callsign_recognition}, que contienen números y pueden confundirse con magnitudes de instrucciones. Por ejemplo, podríamos pensar en la siguiente transcripción del ASR:
    \begin{center}
        \texttt{IBERIA two nine two zero heading}
    \end{center}
    Un posible parseador de magnitudes podría no saber dónde acaba el \textit{callsign} y dónde empieza la magnitud de la instrucción.
    \item \textbf{Múltiples parámetros en la misma instrucción}, como se ilustra que existen en \cite{Zuluaga‑Gomez2023Virtual}. Podríamos pensar en la siguiente transcripción:
    \begin{center}
        \texttt{climb to FL350, turn left heading 090, speed 250}
    \end{center}
    donde tenemos tres parámetros (altura, rumbo y velocidad) para una sola aeronave. 
    \item \textbf{Estructuras temporales encadenadas}, donde se proporcionan instrucciones secuenciales a aeronaves. Es esencial mantener la cohesión temporal en las instrucciones:
    \begin{center}
        \texttt{climb to FL350, after reaching point X descend to 3000}
    \end{center}
    \item \textbf{Normalización}, de las unidades y parámetros de instrucciones. \texttt{FL350}, \texttt{Flight Level three five zero}, \texttt{three five zero} son maneras de decir 35000 pies, y deben interpretarse como tal.
    \item \textbf{Resiliencia ante espontaneidad del discurso}, ya que instrucciones que involucran correcciones de ellas mismas deben ser interpretadas como si fueran instrucciones normales.
\end{enumerate}

\subsubsection{Enfoques actuales y soluciones}
\begin{enumerate}
\item \textbf{A Virtual Simulation-Pilot Agent for Training of Air Traffic Controllers} \cite{Zuluaga-Gomez2023Virtual} 

Se define un componente de \textit{high level parsing} y se le llama \textit{High Level Entity Parser} (HLEP). Cada una de las instrucciones transcritas se utiliza como entrada a este módulo, donde se extraen los campos comunes a todas las instrucciones ATC. Cada una de las entidades que se pueden extraer (en forma de valores etiquetados) se llama \textit{Named entity}. El reconocimiento de las \textit{Named entities} recibe el nombre de \textit{Named Entity Recognition}:

\begin{itemize}
    \item \textit{Callsign} (por ejemplo \texttt{Lima Echo Sierra 3 3 5} $\rightarrow$ \texttt{LES-335})
    \item Comando del ATCO (mantener una altitud, orientarse con determinado ángulo...)
    \item Valores de la instrucción
\end{itemize}

De esta manera, una transcripción del tipo

\texttt{ryanair nine two bravo quebec turn right heading zero nine zero}

se transforma en una instrucción etiquetada, de la forma

\texttt{<callsign>}\textcolor{blue}{\texttt{ryanair nine two bravo quebec}}\texttt{</callsign>} 

\texttt{<command>}\textcolor{red}{\texttt{turn right heading}}\texttt{</command>}

\texttt{<value>}\textcolor{green}{\texttt{zero nine zero}}\texttt{</value>}

Para esta tarea se emplea un Modelo de Lenguaje pre-entrenado y se sigue la estrategia de ajuste fino para la NER. El modelo utilizado fue BERT (\textit{Bidirectional Encoder Representations From Transformers}) (específicamente la versión pre-enetrenada \textbf{BERT-base-uncased}).

Se ajustó a la tarea utilizando el corpus ATCO2 \cite{ATCO2_EndToEndCallsignRecognition2021}. Este \textit{dataset} es altamente conveniente para el \textit{fine-tuning} pues cuenta con transcripciones ya etiquetadas.

En términos de los resultados, el sistema basado en \textbf{BERT} consiguió un resultado de precisión del \textbf{97,5\%} en detección de \textit{callsigns}, mientras que consiguió una del \textbf{82\%} en \textit{commands} y del \textbf{87,2\%} en \textit{values}.

\item \textbf{Research on the Method of Air Traffic Control Instruction Keyword Extraction Based on the Roberta-Attention-BiLSTM-CRF Model} \cite{aerospace12050376}



\end{enumerate}

\section{FlightGear, simulador de vuelo}

\newpage

\section{Related work}
One of the aspects of ATC training involves the \textbf{simulation} of training scenarios.

Current training sessions for ATC consist of two individuals that communicate with each other \cite{eurocontrol:remote-piloting-2021}.
The first and most obvious one is the aspiring ATC in training, that communicates their instructions to solve the initial situation of the aforementioned training scenario.

Our focus is centered in the second individual: the \textbf{pseudo-pilot}. They are in charge
of executing the instructions the aspiring ATC communicated using scenario simulation tools.

As we have already mentioned, the main case of use of a pseudo-pilot is \textbf{ATC training}.
Nevertheless, we have recently seen a rising tendency to use it in other contexts, such as
the validation of novel techniques for their use in real ATC. A pseudo-pilot is the \textit{stand-in pilot},
listening, executing and responding to the ATC's instructions. Furthermore, they may participate
by simulating multiple aircrafts at the same time.

Generally speaking, every conventional pseudo-pilot system follows a working schema like the one described in
Figure~\ref{fig:pseudopiloto-convencional}

\begin{figure}[]
	\centering
	\includegraphics[width=1\linewidth]{assets/conventional-atc-diagram/traditional-pseudopilot.drawio.png}
	\caption{Schema of a conventional pseudopilot. The ATC emits instructions via voice communication channels and the pseudopilot executes them in the simulator. Finally, the simulation framework returns the feedback to the aspiring ATC.}
	\label{fig:pseudopiloto-convencional}
\end{figure}

A traditional pseudo-pilot typically includes a \textbf{Pseudo-pilot working position} which includes all pseudo-piloting tools. It's an accessible interface that allows users to control one or more aircrafts simultaneously, as well as edit the flight plans or other aspects of the simulation.

A Voice Communication System is also in place, allowing the pseudo-pilot to communicate with the trainee ATC using realistic aviation phraseology. Lastly, the \textbf{traffic generation software} generates traffic and complex situations automatically for their resolution by the ATC.

In order to understand the context in which our proposal makes a difference, it is crucial to comprehend the current status of the pseudo-piloting industry.
In more broad terms, we need to understand what the industry currently offers regarding ATC training. These systems define the industry standard and the tools that are used
today by human pseudo-pilots. Table~\ref{tab:simuladores_pp} offers a comparative view of these tools.

\begin{center}
	\begin{longtable}{>{\bfseries}p{3.0cm}|p{3.0cm}|p{3.0cm}|p{4.5cm}}
		\caption{Pseudo-pilot simulators summary}
		\label{tab:simuladores_pp}
		\\
		\toprule
		\textbf{Simulator (Provider)} & \textbf{Primary focus}                      & \textbf{Pseudo-pilot module}        & \textbf{Characteristics}                                                                                      \\
		\midrule
		\endfirsthead
		\multicolumn{4}{c}%
		{{\bfseries \tablename\ \thetable{} -- Continuation of Pseudo-pilot simulators summary}}                                                                                                                                          \\
		\toprule
		\textbf{Simulator (Provider)} & \textbf{Primary focus}                      & \textbf{Pseudo-pilot module}        & \textbf{Characteristics}                                                                                      \\
		\midrule
		\endhead
		\bottomrule
		\endfoot
		\bottomrule
		\endlastfoot

		ESCAPE (Eurocontrol)          & Training and Investigation                  & PWP                                 & Simultaneous control of multiple aircrafts. Advanced radar simulator.                                         \\
		\midrule
		BEST (MicroNav)               & Industry Standard                           & Integrated                          & \textbf{3D and 2D} simulation of aircrafts and \textbf{atmospheric conditions}.                               \\
		\midrule
		CSSOFT ATC Simulator (CSSOFT) & Step minimizing                             & Designed to \textbf{minimize steps} & Allows \textbf{as many flights as possible}. Advanced \textbf{syntax resilience} (minimizes operator errors). \\
		\midrule
		ATCTrSim (HAVELSAN)           & Training                                    & Intuitive and accessible UI         & Prioritizes \textbf{ease of use} for pseudo-piloting in training scenarios.                                   \\
		\midrule
		MaxSim (ADACEL)               & Complete ATC Simulation                     & \textbf{PWP}                        & Simulations use \textbf{real airports}.                                                                       \\
		\midrule
		Indra Simulator (Indra)       & Allows tower, approach and en-route control & \textbf{PWP}                        & 2D and 3D training. Multi-scenario interface.                                                                 \\
		\midrule
		SERA (ASTi)                   & Realistic communications                    & \textbf{Artificial Intelligence}    & Utilizes AI systems for traffic management within the simulator and phraseology reinforcement.                \\
	\end{longtable}
\end{center}

As we can see, solutions focus mainly on the features of each simulation software (2D and 3D, real airport usage, unlimited simultaneous flights...). The dominating paradigm in the industry revolves around the Pseudopilot Working Position. Tools like ESCAPE, MaxSim, or Indra's simulator are, essentially, advanced human interfaces that serve the purpose of providing a comfortable and realistic ATC training interface.
Even more modern AI-using tools like SERA apply it to manage communications, and not for the execution of the ATC instructions. This Commercial State of the Art analysis shows that the current industry depends on a human agent for command execution. None of these platforms offers the ideal execution by voice, essentially removing the
need for a human pseudopilot. This hints at the opportunity of the integration we propose.

Although commercial solutions have not made any approaches that would resemble our proposal, the scientific literature has made some advances in the task of the pseudo-pilot automatization.

The idea of a completely autonomous pseudo-pilot system is not entirely novel. In 2005, Bolczak et al. \cite{bolczak2005accommodating} already proposed a proof of concept that integrated primitive ASR and TTS modules.
This early work was crucial, as it demonstrated a theoretical viability in automatizing the pseudo-pilot tasks completely. It did not become a standard due to the technical limitations of its time.

Recently, advances in the technologies used in \cite{bolczak2005accommodating} have caused a resurgence in the interest to automatize the pseudo-pilot. Lin et al. \cite{LinEtAl} propose the start of a more modular approach, where we firstly focus on
response generation for a pseudopilot. This would cause the human pseudo-pilot to only have to fulfill the ATC's instructions onto the flight simulation interface.

Later research by Zuluaga-Gomez et al. \cite{Zuluaga-Gomez2023Virtual} provides a proof of concept that shows the automatization of the pseudo-pilot replies is possible. This means the pseudo-pilot would need to do less work, but it doesn't quite automate the full process. This begins to close the gap between the full automatization and the current state of the commercial systems.

\subsection{Literatura y soluciones alternativas}
\begin{itemize}
	\item \cite{Zuluaga-Gomez2023Virtual}
	      proporciona una prueba de concepto que demuestra que se puede cambiar el paradigma actual a uno de (por lo menos) automatización de la respuesta del controlador. Esto supone una liberación de carga, pero no la automatización del proceso completo. Las tasas de error en sistemas como ASR y construcción de respuestas (WER) evolucionan cada vez más a un nivel aceptable.
\end{itemize}

En definitiva, se ha llegado a un modelo como lo que se puede ver en la Figura~\ref{fig:pseudopiloto-respuesta-convencional}. Proporciona los procesos de automatización de respuesta, pero no de ejecución de instrucciones.

\begin{figure}[htbp]
	\centering
	\includesvg[width=1\textwidth]{assets/diagram_semiautomated/diagrama.svg}
	\caption{Esquema semiautomatizado propuesto para el proceso de pseudopilotaje, integrando módulos automáticos y respuesta supervisada.}
	\label{fig:pseudopiloto-respuesta-convencional}
\end{figure}

\begin{figure}[htbp]
	\centering
	\includegraphics[width=1\textwidth]{assets/pseudopilot-semithere.png}
	\caption{Esquema semiautomatizado propuesto para el proceso de pseudopilotaje, integrando módulos automáticos y respuesta supervisada.}
	\label{fig:pseudopiloto-respuesta-convencional}
\end{figure}

\subsection{Propuesta actual}
Nosotros proponemos un modelo que da un paso más, replicando el diagrama propuesto en la Figura~\ref{fig:pseudopiloto-full-auto}

\begin{figure}[htbp]
	\centering
	\includegraphics[width=1\textwidth]{assets/pseudo-full-auto.png}
	\caption{Esquema de pseudopilotaje completamente automatizado: el sistema recibe transmisiones de voz del controlador, las procesa mediante módulos ASR, comprensión de instrucciones y ejecuta acciones en el simulador de vuelo de forma autónoma, generando respuestas (TTS) sin intervención humana.}
	\label{fig:pseudopiloto-full-auto}
\end{figure}

Esto proporciona una eliminación del componente humano, proporcionando ciertas ventajas:
\begin{enumerate}
	\item Menor requerimiento de esfuerzo humano para entrenar ATCs.
	\item Mayor capacidad para escalar las simulaciones, permitiendo situaciones de control aéreo de un gran número de aeronaves sin apenas esfuerzo.
	\item Plausible autonomía del pseudopiloto para complicar escenarios de control aéreo.
	\item Mecanismo de validación automática para métodos de control aéreo.
\end{enumerate}
\clearpage
\printbibliography
\end{document}
