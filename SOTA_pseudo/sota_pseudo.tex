\section{Estudio del \textit{State of the Art} en pseudopilotos}

\subsection{Definición del problema: ¿qué es un pseudopiloto?}
Existe una parte práctica de entrenamiento para el control aéreo que consiste en \textbf{simular} escenarios de entrenamiento.

En los ejercicios de entrenamiento para control aéreo existen dos individuos que se comunican~\cite{eurocontrol:remote-piloting-2021}.
El primero y más obvio es el aspirante a controlador aéreo, que comunica sus instrucciones para resolver la situación inicial del escenario de entrenamiento.

El segundo es en quien centramos nuestro interés: el \textbf{pseudopiloto}. Se encarga de ejecutar las instrucciones del controlador aéreo a través de herramientas de simulación de escenarios.

Como ya hemos mencionado, su principal uso es el de entrenamiento, aunque recientemente vemos una aparición más creciente en otros contextos, como la validación de novedades técnicas para su uso
en control aéreo real. Un pseudopiloto hace las `veces de piloto', escuchando, ejecutando y respondiendo al controlador aéreo, incluso haciéndose cargo de la simulación de múltiples aeronaves.

En general, todo sistema de pseudopilotos convencional sigue un esquema de funcionamiento como el que se ve en la Figura~\ref{fig:pseudopiloto-convencional}

\begin{figure}[htbp]
    \centering
    \includegraphics[width=0.75\linewidth]{assets/pseudo-basico.png}
    \caption{Esquema de funcionamiento de un pseudopiloto convencional: el controlador emite instrucciones, el pseudopiloto las ejecuta en el simulador y el entorno devuelve la retroalimentación.}
\label{fig:pseudopiloto-convencional}
\end{figure}



\subsection{Pseudopilotos tradicionales}

\begin{itemize}
  \item \textbf{ESCAPE (Eurocontrol)} \\
  Incluye el módulo \textit{Pilot Working Position (PWP)}. Permite el control simultáneo de múltiples aeronaves con comandos de altitud, rumbo y velocidad. Soporta operación remota y está diseñado para simulaciones de radar avanzadas.

  \item \textbf{PEGASUS (Eurocontrol Experimental Centre)} \\
  Herramienta de investigación en gestión del tráfico aéreo. Integra interfaces para pseudopilotos en escenarios complejos y validación de procedimientos.

  \item \textbf{Target Generation Facility (TGF, FAA)} \\
  Utiliza el módulo \textit{SimPilot}. Ofrece interfaces específicas para cada fase del vuelo (salida, ruta, llegada). Ampliamente usado en centros de simulación de la FAA.

  \item \textbf{ROSE (DFS Alemania)} \\
  Sistema de simulación ATC que incorpora estaciones de pseudopiloto con comandos rápidos para maniobras básicas y avanzadas.

  \item \textbf{ANSART Pilot Simulator} \\
  Plataforma comercial con GUI simplificada, integración modular, y capacidad de preprogramar vuelos. Adecuado para academias privadas y ANSP.

  \item \textbf{MicroNav BEST (Beginning to End for Simulation and Training)} \\
  Sistema versátil con módulos de torre, radar y pseudopiloto. Permite personalizar escenarios y comandos por interfaz gráfica.

  \item \textbf{SACTA SIM (ENAIRE / Indra)} \\
  Simulador ATC usado en España. Los pseudopilotos acceden a una interfaz conectada a la base de datos de tráfico simulada, con comandos compatibles con SACTA.

  \item \textbf{ATCoach / SkyRadar} \\
  Plataformas educativas que permiten control básico de aeronaves simuladas. Útiles en universidades y centros de formación inicial.
\end{itemize}


\subsection{Literatura reciente}